% Options for packages loaded elsewhere
\PassOptionsToPackage{unicode}{hyperref}
\PassOptionsToPackage{hyphens}{url}
%
\documentclass[
]{article}
\usepackage{lmodern}
\usepackage{amssymb,amsmath}
\usepackage{ifxetex,ifluatex}
\ifnum 0\ifxetex 1\fi\ifluatex 1\fi=0 % if pdftex
  \usepackage[T1]{fontenc}
  \usepackage[utf8]{inputenc}
  \usepackage{textcomp} % provide euro and other symbols
\else % if luatex or xetex
  \usepackage{unicode-math}
  \defaultfontfeatures{Scale=MatchLowercase}
  \defaultfontfeatures[\rmfamily]{Ligatures=TeX,Scale=1}
\fi
% Use upquote if available, for straight quotes in verbatim environments
\IfFileExists{upquote.sty}{\usepackage{upquote}}{}
\IfFileExists{microtype.sty}{% use microtype if available
  \usepackage[]{microtype}
  \UseMicrotypeSet[protrusion]{basicmath} % disable protrusion for tt fonts
}{}
\makeatletter
\@ifundefined{KOMAClassName}{% if non-KOMA class
  \IfFileExists{parskip.sty}{%
    \usepackage{parskip}
  }{% else
    \setlength{\parindent}{0pt}
    \setlength{\parskip}{6pt plus 2pt minus 1pt}}
}{% if KOMA class
  \KOMAoptions{parskip=half}}
\makeatother
\usepackage{xcolor}
\IfFileExists{xurl.sty}{\usepackage{xurl}}{} % add URL line breaks if available
\IfFileExists{bookmark.sty}{\usepackage{bookmark}}{\usepackage{hyperref}}
\hypersetup{
  pdftitle={Primer Parcial (30pts). Solucionario},
  pdfauthor={Lic. Alvaro Chirino Gutierrez},
  hidelinks,
  pdfcreator={LaTeX via pandoc}}
\urlstyle{same} % disable monospaced font for URLs
\usepackage[margin=1in]{geometry}
\usepackage{color}
\usepackage{fancyvrb}
\newcommand{\VerbBar}{|}
\newcommand{\VERB}{\Verb[commandchars=\\\{\}]}
\DefineVerbatimEnvironment{Highlighting}{Verbatim}{commandchars=\\\{\}}
% Add ',fontsize=\small' for more characters per line
\usepackage{framed}
\definecolor{shadecolor}{RGB}{248,248,248}
\newenvironment{Shaded}{\begin{snugshade}}{\end{snugshade}}
\newcommand{\AlertTok}[1]{\textcolor[rgb]{0.94,0.16,0.16}{#1}}
\newcommand{\AnnotationTok}[1]{\textcolor[rgb]{0.56,0.35,0.01}{\textbf{\textit{#1}}}}
\newcommand{\AttributeTok}[1]{\textcolor[rgb]{0.77,0.63,0.00}{#1}}
\newcommand{\BaseNTok}[1]{\textcolor[rgb]{0.00,0.00,0.81}{#1}}
\newcommand{\BuiltInTok}[1]{#1}
\newcommand{\CharTok}[1]{\textcolor[rgb]{0.31,0.60,0.02}{#1}}
\newcommand{\CommentTok}[1]{\textcolor[rgb]{0.56,0.35,0.01}{\textit{#1}}}
\newcommand{\CommentVarTok}[1]{\textcolor[rgb]{0.56,0.35,0.01}{\textbf{\textit{#1}}}}
\newcommand{\ConstantTok}[1]{\textcolor[rgb]{0.00,0.00,0.00}{#1}}
\newcommand{\ControlFlowTok}[1]{\textcolor[rgb]{0.13,0.29,0.53}{\textbf{#1}}}
\newcommand{\DataTypeTok}[1]{\textcolor[rgb]{0.13,0.29,0.53}{#1}}
\newcommand{\DecValTok}[1]{\textcolor[rgb]{0.00,0.00,0.81}{#1}}
\newcommand{\DocumentationTok}[1]{\textcolor[rgb]{0.56,0.35,0.01}{\textbf{\textit{#1}}}}
\newcommand{\ErrorTok}[1]{\textcolor[rgb]{0.64,0.00,0.00}{\textbf{#1}}}
\newcommand{\ExtensionTok}[1]{#1}
\newcommand{\FloatTok}[1]{\textcolor[rgb]{0.00,0.00,0.81}{#1}}
\newcommand{\FunctionTok}[1]{\textcolor[rgb]{0.00,0.00,0.00}{#1}}
\newcommand{\ImportTok}[1]{#1}
\newcommand{\InformationTok}[1]{\textcolor[rgb]{0.56,0.35,0.01}{\textbf{\textit{#1}}}}
\newcommand{\KeywordTok}[1]{\textcolor[rgb]{0.13,0.29,0.53}{\textbf{#1}}}
\newcommand{\NormalTok}[1]{#1}
\newcommand{\OperatorTok}[1]{\textcolor[rgb]{0.81,0.36,0.00}{\textbf{#1}}}
\newcommand{\OtherTok}[1]{\textcolor[rgb]{0.56,0.35,0.01}{#1}}
\newcommand{\PreprocessorTok}[1]{\textcolor[rgb]{0.56,0.35,0.01}{\textit{#1}}}
\newcommand{\RegionMarkerTok}[1]{#1}
\newcommand{\SpecialCharTok}[1]{\textcolor[rgb]{0.00,0.00,0.00}{#1}}
\newcommand{\SpecialStringTok}[1]{\textcolor[rgb]{0.31,0.60,0.02}{#1}}
\newcommand{\StringTok}[1]{\textcolor[rgb]{0.31,0.60,0.02}{#1}}
\newcommand{\VariableTok}[1]{\textcolor[rgb]{0.00,0.00,0.00}{#1}}
\newcommand{\VerbatimStringTok}[1]{\textcolor[rgb]{0.31,0.60,0.02}{#1}}
\newcommand{\WarningTok}[1]{\textcolor[rgb]{0.56,0.35,0.01}{\textbf{\textit{#1}}}}
\usepackage{graphicx,grffile}
\makeatletter
\def\maxwidth{\ifdim\Gin@nat@width>\linewidth\linewidth\else\Gin@nat@width\fi}
\def\maxheight{\ifdim\Gin@nat@height>\textheight\textheight\else\Gin@nat@height\fi}
\makeatother
% Scale images if necessary, so that they will not overflow the page
% margins by default, and it is still possible to overwrite the defaults
% using explicit options in \includegraphics[width, height, ...]{}
\setkeys{Gin}{width=\maxwidth,height=\maxheight,keepaspectratio}
% Set default figure placement to htbp
\makeatletter
\def\fps@figure{htbp}
\makeatother
\setlength{\emergencystretch}{3em} % prevent overfull lines
\providecommand{\tightlist}{%
  \setlength{\itemsep}{0pt}\setlength{\parskip}{0pt}}
\setcounter{secnumdepth}{-\maxdimen} % remove section numbering

\title{Primer Parcial (30pts). Solucionario}
\usepackage{etoolbox}
\makeatletter
\providecommand{\subtitle}[1]{% add subtitle to \maketitle
  \apptocmd{\@title}{\par {\large #1 \par}}{}{}
}
\makeatother
\subtitle{Programacion Estadística II}
\author{Lic. Alvaro Chirino Gutierrez}
\date{8/6/2020}

\begin{document}
\maketitle

\hypertarget{pregunta-1-5pts.}{%
\subsection{Pregunta 1 (5pts).}\label{pregunta-1-5pts.}}

\begin{itemize}
\tightlist
\item
  Liste y comente las fases del descubrimiento del conocimiento en las
  bases de datos (KDD)
\end{itemize}

\textbf{Resp.}

\begin{enumerate}
\def\labelenumi{\arabic{enumi}.}
\tightlist
\item
  Selección: Identificar los datos de interés
\item
  Preprocesado: Trabajar con la base
\item
  Transformacion: Adecuar la base segun la naturaleza de las variables
\item
  Mineria de datos: Aplicar el modelo adecuado según las necesidades
\item
  Interpretacion/Evaluación: Evaluar los resultados, retroalimentación
\end{enumerate}

\begin{itemize}
\tightlist
\item
  Describa que es un warehouse
\end{itemize}

\textbf{Resp.} Es un almancen de datos que permite gestionar las bases
de datos en su múltiples formatos

\begin{itemize}
\tightlist
\item
  Describa la diferencia entre componentes principales y análisis de
  correspondencia
\end{itemize}

\textbf{Resp.} Componentes principales: Aplicación de la descomposición
espectral para la transformacion de variables, Análisis de
Correspondencia: Técnica gráfica que aplica una transformacion SVD a
partir de una matriz de contingencias de variables cualitativas.

\begin{itemize}
\tightlist
\item
  Describa la diferencia entre k-center y los clusters jerárquicos
\end{itemize}

\textbf{Resp.} Los métodos k-center parten de un valor k prefijado, los
métodos jerárquicos no requieren un valor previo de k.

\begin{itemize}
\tightlist
\item
  En el proceso de imputar si una variable alcanza un índice influx
  alto, esto siginifica que: ¿esta variable es más útil para imputar
  otras variables?. Si, No, explique.
\end{itemize}

\textbf{Resp.} NO, el influx alto significa que dicha variable sera más
facil de imputar.

\hypertarget{pregunta-2-5-pts}{%
\subsection{Pregunta 2 (5 pts):}\label{pregunta-2-5-pts}}

Empleando la encuesta a hogares 2018, obtenga una tabla que contenga el
porcentaje y total de hogares (expandido) por departamento y área, que
tengan a algún miembro del hogar con diabetes ó hipertension arterial ó
enfermedad del corazón.

\begin{Shaded}
\begin{Highlighting}[]
\KeywordTok{load}\NormalTok{(}\KeywordTok{url}\NormalTok{(}\StringTok{"https://github.com/AlvaroLimber/EST-384/raw/master/data/eh18.Rdata"}\NormalTok{))}
\end{Highlighting}
\end{Shaded}

\hypertarget{pregunta-3-10-pts}{%
\subsection{Pregunta 3 (10 pts)}\label{pregunta-3-10-pts}}

\begin{itemize}
\tightlist
\item
  PCA: Empleando la base de datos de la encuesta a hogares 2018,
  seleccione solo al jefe del hogar y calcule el PCA con las variables;
  mujer, edad, años de educación, horas trabajadas a la semana, ingreso
  laboral, ingreso no laboral, ingreso percapita del hogar. Identifique
  la cantidad de componentes a retener con el criterio de eigenvalores
  que superen la unidad. Calcule los componentes principales retenidos y
  grafique sus histogramas.
\item
  CA: Empleando la base de datos de las ENDSA para el año 2008, para las
  10 variables de violencia tranformar a binarias como (1=frecuentemente
  ó a veces 0=Nunca) y realizar el MCA para las 10 variables de
  violencia transformadas incluyendo la variable sexo. Comentar los
  resultados.
\end{itemize}

\hypertarget{pregunta-4-10-pts}{%
\subsection{Pregunta 4 (10 pts)}\label{pregunta-4-10-pts}}

Clustering: Empleando la base de datos de las elecciones del 20 de
octubre, aplique de forma separada para los municipios y los países, en
términos relativos sin considerar los blanco y nulos:

\begin{itemize}
\tightlist
\item
  El Cluster k-center, identifique el mejor valor de \(k\) entre 2 al
  10, el mejor centro; media, mediana o medoide, y la mejor distancia;
  euclideana, manhattan
\item
  El Cluster jerarquico, identifique el mejor valor de \(k\) entre 2 al
  10, el mejor método; complete, single, average y la mejor distancia;
  euclideana, manhattan
\end{itemize}

\hypertarget{pregunta-5-opcional-5-pts}{%
\subsection{Pregunta 5 (Opcional, 5
pts):}\label{pregunta-5-opcional-5-pts}}

Programe una función que calcule el coeficiente de silueta dada una
matriz de distancias y un vector de identificación del cluster

\end{document}
